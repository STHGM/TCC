%%%%%%%%%%%%%%%%%%%%%%%%%%%%%%%%%%%%%%%%%%%%%%%%%%%%%%%%%%%%%%%%%%%%%%%%%%%%%%%
% CAP�TULO 5
\chapter{RESULTADOS E DISCUSS�ES}  

Espera-se que o uso do estilo de formata\c{c}\~ao \LaTeX\ adequado \`as Normas para Elabora\c{c}\~ao de Trabalhos Acad\^emicos da UTFPR ({\ttfamily normas-utf-tex.cls}) facilite a escrita de documentos no \^ambito desta institui\c{c}\~ao e aumente a produtividade de seus autores. Para usu\'arios iniciantes em \LaTeX, al\'em da bibliografia especializada j\'a citada, existe ainda uma s\'erie de recursos~\cite{CTAN2009} e fontes de informa\c{c}\~ao~\cite{TeX-Br2009,Wikibooks2009} dispon\'iveis na Internet.

Recomenda-se o editor de textos Kile como ferramenta de composi\c{c}\~ao de documentos em \LaTeX\ para usu\'arios Linux. Para usu\'arios Windows recomenda-se o editor \TeX nicCenter~\cite{TeXnicCenter2009}. O \LaTeX\ normalmente j\'a faz parte da maioria das distribui\c{c}\~oes Linux, mas no sistema operacional Windows \'e necess\'ario instalar o software \textsc{MiK}\TeX~\cite{MiKTeX2009}.

Al\'em disso, recomenda-se o uso de um gerenciador de refer\^encias como o JabRef~\cite{JabRef2009} ou Mendeley~\cite{Mendeley2009} para a cataloga\c{c}\~ao bibliogr\'afica em um arquivo \textsc{Bib}\TeX, de forma a facilitar cita\c{c}\~oes atrav\'es do comando {\ttfamily \textbackslash cite\{\}} e outros comandos correlatos do pacote \textsc{abn}\TeX. A lista de refer\^encias deste documento foi gerada automaticamente pelo software \LaTeX\ + \textsc{Bib}\TeX\ a partir do arquivo {\ttfamily reflatex.bib}, que por sua vez foi composto com o gerenciador de refer\^encias JabRef.

O estilo de formata\c{c}\~ao \LaTeX\ da UTFPR e este exemplo de utiliza\c{c}\~ao foram elaborados por Diogo Rosa Kuiaski (diogo.kuiaski@gmail.com) e Hugo Vieira Neto (hvieir@utfpr.edu.br), com contribui\c{c}\~oes de C\'esar Vargas Benitez. Sugest\~oes de melhorias s\~ao bem-vindas.
